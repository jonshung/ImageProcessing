\documentclass[12pt, a4paper]{extarticle}
\usepackage[utf8]{inputenc}
\usepackage[T4]{fontenc}
\usepackage[vietnamese]{babel}
\usepackage{amsmath,amsfonts,amssymb,graphicx,geometry,enumerate,enumitem,tikz,tkz-tab,multicol,fancybox,hyperref,lscape,pdflscape,adjustbox,float}
\usepackage{multirow}
\usepackage{mathtools}
\usepackage{tabularray}
\usepackage{tabularx}
\usepackage{enumerate}% http://ctan.org/pkg/enumerate
\DeclareMathOperator*{\modd}{\scriptstyle\%}

\geometry{  
    a4paper,
    top=2cm,
    left=1.5cm,
    right=1.5cm,
}

\usepackage[document]{ragged2e}
\usepackage{fontawesome}
\usepackage{pbox}
\usepackage{marvosym}
\usepackage{xcolor}
\usepackage{alltt}
\usepackage{array}

\usepackage{listings}
\definecolor{ao(english)}{rgb}{0.0,0.5, 0.0}
\definecolor{airforceblue}{rgb}{0.36,0.54, 0.66}
\lstdefinestyle{cpp}{
    language=C++,
    basicstyle=\ttfamily,
    keywordstyle=\color{blue}\bfseries,
    commentstyle=\color{ao(english)},
    stringstyle=\color{red},
    showstringspaces=false,
    breaklines=true,
    breakatwhitespace=true,
    tabsize=4,
    moredelim=[is][\color{airforceblue}\bfseries]{@}{@}
}

\lstdefinestyle{mypythonstyle}{
    %backgroundcolor=\color{white},   % choose the background color
    basicstyle=\ttfamily\footnotesize,% the size of the fonts that are used for the code
    breakatwhitespace=false,         % sets if automatic breaks should only happen at whitespace
    breaklines=true,                 % sets automatic line breaking
    captionpos=b,                    % sets the caption-position to bottom
    commentstyle=\color{ao(english)},% comment style
    deletekeywords={...},            % if you want to delete keywords from the given language
    escapeinside={\%*}{*)},          % if you want to add LaTeX within your code
    extendedchars=true,              % lets you use non-ASCII characters
    %frame=single,                    % adds a frame around the code
    keepspaces=true,                 % keeps spaces in text, useful for keeping indentation of code (possibly needs columns=flexible)
    keywordstyle=\color{blue},       % keyword style
    language=Python,                 % the language of the code
    morekeywords={*,...},            % if you want to add more keywords to the set
    %numbers=left,                    % where to put the line-numbers; possible values are (none, left, right)
    %numbersep=5pt,                   % how far the line-numbers are from the code
    %numberstyle=\tiny\color{mygray}, % the style that is used for the line-numbers
    rulecolor=\color{black},         % if not set, the frame-color may change on line-breaks within not-black text (e.g. comments (green here))
    showspaces=false,                % show spaces everywhere adding particular underscores; it overrides 'showstringspaces'
    showstringspaces=false,          % underline spaces within strings only
    showtabs=false,                  % show tabs within strings adding particular underscores
    stepnumber=2,                    % the step between two line-numbers. If it's 1, each line will be numbered
    stringstyle=\color{red},         % string literal style
    tabsize=2,                       % sets default tabsize to 2 spaces
    title=\lstname                   % show the filename of files included with \lstinputlisting; also try caption instead of title
}

\lstset{style=cpp}

\usepackage{titlesec}

% Define new section format for "Chương x" numbering
\titleformat{\section}{\Large\bfseries}{Chương \thesection}{1em}{}
\titleformat{\subsection}{\large\bfseries}{\arabic{subsection}}{1em}{}
\titleformat{\subsubsection}{\normalsize\bfseries}{\thesection.\arabic{subsubsection}}{1em}{}

% Set the numbering styles
\renewcommand{\thesection}{\arabic{section}}
\renewcommand{\thesubsection}{\arabic{subsection}}
\renewcommand{\thesubsubsection}{\thesection.\arabic{subsubsection}}

% Reset subsection counter at each new section
\makeatletter
\@addtoreset{subsection}{section}
\@addtoreset{subsubsection}{subsection}
\makeatother

\usepackage{mdframed}

\newmdenv[linecolor=black,skipabove=\topsep,skipbelow=\topsep,
leftmargin=-5pt,rightmargin=-5pt,
innerleftmargin=5pt,innerrightmargin=5pt]{mybox} % frame box

\usepackage{fancyhdr}
\pagestyle{fancy}
\lhead{}
\rhead{}
\renewcommand{\headrulewidth}{0pt}
\lfoot{\texttt{VNUHCM-US-FIT}}
\rfoot{\small\texttt{2212102-22127134-22127318-21127402}}
\renewcommand{\footrulewidth}{1pt}

\renewcommand{\contentsname}{Contents}

\begin{document}


    


    \begin{titlepage}
        \begin{center}
            {
                \Large
                \textbf{Vietnam National University Ho Chi Minh City}
            }

            \vspace{0.5cm}
            {
                \Large
                \textbf{University of Science}
            }

            \vspace{0.5cm}
            {
                \large
                \textit{Faculty of Information Technology}
            }

            \vspace{2cm}
            \includegraphics[width=0.4\textwidth]{hcmus.jpg}

            \vspace{2cm}
            {
                \huge
                \textbf{Digital Image and Video Processing:} \vspace{0.3cm} \\
                \textbf{BACKGROUND REMOVAL}
            }

            \vspace{3cm}
            {

                \begin{tabular}{l l}
                    Course & {\large\textsc{CSC16005 - Digital Image and Video Processing}}\vspace{0.2cm}\\
                    Group & {\large DaBalance}\vspace{0.2cm}\\
                    Lecturers & \textbf{Lý Quốc Ngọc}\\
                     & \textbf{Nguyễn Mạnh Hùng}\\
                     & \textbf{Phạm Thanh Tùng}\vspace{0.2cm}\\
                    Participants 
                    & \textbf{22127102} $-$ \textbf{Vũ Gia Hân}\\
                    & \textbf{22127134} $-$ \textbf{Ngũ Kiệt Hùng} \\
                    & \textbf{22127318} $-$ \textbf{Trang Minh Nhựt} \\
                    & \textbf{22127402} $-$ \textbf{Bế Lã Anh Thư} 
                \end{tabular}
            }
            \vfill
            \vspace{0.5cm}
            {
                \textit{HCMC, 2023}
            }
        \end{center}
    \end{titlepage}
    \pagebreak



    \begin{tabular}{p{0.25\linewidth} p{0.75\linewidth}}
        & \emph{For a full experience, use a PDF reader that visually indicate the words associated with hyperlink.}\\
    \end{tabular}
    \tableofcontents
    \pagebreak

    \section{Giới thiệu}
        \subsection{Ý nghĩa về khoa học của chủ đề}
        \subsection{Ý nghĩa về ứng dụng của chủ đề}
        \subsection{Phát biểu bài toán}
        \subsection{Đóng góp của báo cáo}

    \section{Các công trình nghiên cứu liên quan}
        \subsection{Chọn lọc các công trình nghiên cứu liên quan đến chủ đề}
        \subsection{Trình bày quá trình phát triển của các giải pháp liên quan đến chủ đề}
        \subsection{Lập bảng so sánh các giải pháp dựa trên một số tiêu chí tự chọn}

    \section{Đánh giá thành viên}
        \begin{center}
            \begin{tabular}{|c|l|c|l|} \hline
    
                \textbf{Member} &	\textbf{Assignments} & \textbf{DoC} & \textbf{Comments} \\      \hline
    
                 & &	100\% & None\\   \hline 
                 & &	100\% & None\\   \hline 
                 & &  100\% & None\\ \hline
                 & &  100\% & None\\ \hline
            \end{tabular}
        \end{center}          


    \section{Tài liệu tham khảo}
    \begin{enumerate} [label = $-$]
        \item Raylib:
            \begin{enumerate} [label = {[\arabic*]}]
                \item \href{https://electronstudio.github.io/raylib-python-cffi/pyray.html}{Raylib Python}.
                \item \href{https://github.com/raysan5/raylib/blob/master/examples/textures/textures_fog_of_war.c}{Fog of war}.
                \item \href{https://www.raylib.com/cheatsheet/cheatsheet.html}{Raylib - CheatSheet}.
            \end{enumerate}
            
        \item Hide and Seek Algorithms: 
            \begin{enumerate} [label = {[\arabic*]}]
                \item \href{https://en.wikipedia.org/wiki/Bresenham%27s_line_algorithm?fbclid=IwAR3xf8SQy0Ita7FKyIvI_geSLWZgar7R-uT3c2utJ1trZqYf9fKLtEwPuec#Notes}{Bresenham's line algorithm}.
                
            \end{enumerate}
    \end{enumerate}
\end{document}